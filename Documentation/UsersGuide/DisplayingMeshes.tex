\section{Displaying the Mesh}
\label{section:DisplayingMeshes}

Upon successful creation of a mesh, you either need to load it in manually via the \emph{open MOAB file dialog} accessed from the \emph{open MOAB file item} in the \emph{file menu}, or, if you've just meshed an INP file you lodaed in, the mesh will load in automatically.

\subsection{Views of the Mesh}
To control viewing the mesh, be sure to click on the \emph{mesh tab} of the \emph{inputs panel}.  RGG allows you to six different options to control the \emph{3D view} of the mesh, listed below:

\begin{itemize}
	\item{Volumes}
	\item{Boundary}
	\item{Surfaces}
	\item{Neumann Sets}
	\item{Dirichlet Sets}
	\item{Material Sets}
\end{itemize}

More detail on these views is provided below.  Additionally, you can check the \emph{show edges checkbox} to view the mesh superimposed on the 3D view and check the \emph{color checkbox} to colorize the different pieces of the view based on the option you've selected.

\subsubsection{Volumes}
Clicking the \emph{volumes option} shows all the volumes of the mesh.  Checking the color checkbox will colorize all of the distinct volumes in different colors.

\subsubsection{Boundary}
In this mode, RGG displays only the boundary conditions.  Checking the color checkbox will colorize all of the distinct volumes in different colors.

\subsubsection{Surfaces}
When the \emph{surfaces option} is selected, RGG displays only the surfaces of the mesh.

\subsubsection{Neumann Sets}
Clicking the \emph{Neumann Sets option} will show the natural boundary conditions on sides of domains.

\subsubsection{Dirichlet Sets}
Clicking the \emph{Dirichlet Sets option} will show the essential boundary conditions on points of domains.

\subsubsection{Material Sets}
Material sets displays all the volumes of the mesh, but colorizes them on a per material basis.  Additionally, you can toggle the visability of materials by checking and unchecking them in the \emph{materials tab} of the inputs panel.