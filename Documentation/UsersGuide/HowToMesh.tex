\section{Generating a Mesh from an INP file}

Now that the system preferences have been adjusted to include AssyGen, Cubit, and CoreGen, you can generate a mesh by opening an INP file and though the \emph{Run MeshKit RGG dialog} accessed through the \emph{tools menu}.  Note the ``x" icon for the assemblies and core in the inputs pane.  This designates that RGG believes that these components need to be meshed.

\begin{figure}[H]
	\begin{center}
		\includegraphics[width=0.5\linewidth]{Images/mesh-3.png}
		\caption{Opening the Run MeshKit RGG dialog.}
		\label{fig:Mesh3}
	\end{center}
\end{figure}

You may be prompted to save changes before proceeding.

The dialog should be populated with the assemblies that need to be re-meshed automatically.  In the example below, two assemblies need to be meshed, which also means that the core must be remeshed.

\begin{figure}[H]
	\begin{center}
		\includegraphics[width=0.5\linewidth]{Images/mesh-4.png}
		\caption{The Run MeshKit RGG dialog.}
		\label{fig:Mesh4}
	\end{center}
\end{figure}

Note that you can force the meshing of any assembly, or selectively mesh only certain assemblies with the \emph{force process assemblies checkbox} and the \emph{process selected assemblies button}.  To force the processing of the core, check the \emph{force process core checkbox}.  We'll click the \emph{process all button} without changing any defaults to process only the assemblies that need to be meshed and then process the core.  A window that gives you the status of the processing should come up, similar to the one below:

\begin{figure}[H]
	\begin{center}
		\includegraphics[width=0.5\linewidth]{Images/mesh-5.png}
		\caption{Running meshing.}
		\label{fig:Mesh5}
	\end{center}
\end{figure}

You can see the output of MeshKit's AssyGen and CoreGen by clicking on the \emph{view output checkbox}.

To verify that the assemblies and core have been meshed, confirm that the ``x" icons seen previously are now be green squares, as shown in ~\ref{fig:Mesh6}.

\begin{figure}[H]
	\begin{center}
		\includegraphics[width=0.5\linewidth]{Images/mesh-6.png}
		\caption{Verifying successful meshing.}
		\label{fig:Mesh6}
	\end{center}
\end{figure}

Note that the mesh needs to be loaded in after MeshKit creates it.  During this process, the tools menu is inaccessible.