\section{The types of Hexagonal Cores}

RGG has built-in support for a variety of hexagonal-based cores, including full hexagonal cores, $\frac{1}{6}$ hexagonal flat cores, $\frac{1}{6}$ hexagonal vertex cores, and $\frac{1}{12}$ hexagonal cores.  In this example we will construct a full hexagonal core.

\section{Simple Hexagonal Flat Core}

First, we'll need to create a new hexagonal core.

\begin{figure}[H]
	\begin{center}
		\includegraphics[width=0.5\linewidth]{Images/hex-new.png}
		\caption{Select the option to create a new hexagonal core.}
		\label{fig:Hex1}
	\end{center}
\end{figure}

The two panels on the left of the application will change along with the 2D and 3D views.  We'll now see that the inputs panel contains an item named ``Core" with a subitem ``Assy\_0."  The core panel should show a hexagon labeled ``Assy\_0."  The 2D and 3D view will contain the initial core. Confirm that your application looks similar to what's shown in Figure \ref{fig:Hex2}.

\begin{figure}[H]
	\begin{center}
		\includegraphics[width=0.5\linewidth]{Images/hex-new-init.png}
		\caption{The initial view after created a new hex core.}
		\label{fig:Hex2}
	\end{center}
\end{figure}

Next, we'll specify how many layers we'd like our lattice to have.  In the core panel, either type in the number of desired layers or use the buttons to the right of the field to increment and decrement the number.  The panel should look something like what's displayed in Figure \ref{fig:Hex3}.

\begin{figure}[H]
	\begin{center}
		\includegraphics[width=0.5\linewidth]{Images/hex-increase-core-layers.png}
		\caption{Updating the number of layers of the core.}
		\label{fig:Hex3}
	\end{center}
\end{figure}

Be sure to click the apply button below to see apply changes.

\begin{figure}[H]
	\begin{center}
		\includegraphics[width=0.5\linewidth]{Images/hex-apply-button.png}
		\caption{The apply button.}
		\label{fig:Hex4}
	\end{center}
\end{figure}

\section{Configuring Duct}

We now need to configure the duct.  In the inputs panel, select the duct.   This will change the lower panel to show the materials of the duct control panel.  Next select the only segment in the ``Duct Segment" table.  This will populate the ``Material Layers."   For this duct, we'll stay simple and make the material water.  Change the material type to water by clicking the drop-down and selecting ``water."

\begin{figure}[H]
	\begin{center}
		\includegraphics[width=0.5\linewidth]{Images/hex-set-duct-material.png}
		\caption{Selecting water as our material.}
		\label{fig:Hex7}
	\end{center}
\end{figure}

Your screen should now look like Figure \ref{fig:Hex8}.

\begin{figure}[H]
	\begin{center}
		\includegraphics[width=0.5\linewidth]{Images/hex-duct-result.png}
		\caption{Creating a duct.}
		\label{fig:Hex8}
	\end{center}
\end{figure}

\section{Adding a Pin}
\subsection{Creating a Pin}

To create a pin in the assembly, right click on the name of the assembly and select ``Create Pin."

\begin{figure}[H]
	\begin{center}
		\includegraphics[width=0.5\linewidth]{Images/hex-create-pin.png}
		\caption{Creating a pin.}
		\label{fig:Hex9}
	\end{center}
\end{figure}

\subsection{Configuring the Pin}

You can edit the name, label, and cell material of the pin.  Refer to Figure \ref{fig:Hex10} to see how to change them.  We will leave them to the defaults.

\begin{figure}[H]
	\begin{center}
		\includegraphics[width=0.5\linewidth]{Images/hex-default-pin-params.png}
		\caption{Name, label, and cell material for a pin.}
		\label{fig:Hex10}
	\end{center}
\end{figure}

We now need to add pieces of the pin.  There are two kinds of pin pieces: frustums and cylinders.  On creation of the pin, a cylinder piece is added.  This can be changed to frustum using the drop-box.

Then, confirm that our segment type is a cylinder, and that the sum of the length of the segments is equal to the length of the duct (10) (Figure ~\ref{fig:Hex12}).

\begin{figure}[H]
	\begin{center}
		\includegraphics[width=0.5\linewidth]{Images/hex-pin-configure.png}
		\caption{Confirming our cylinder's configuration.}
		\label{fig:Hex12}
	\end{center}
\end{figure}

Now, we're going to modify the material of this cylinder (Figure ~\ref{fig:Hex13}).  We'll make this a control rod, so we'll change the material to match.

\begin{figure}[H]
	\begin{center}
		\includegraphics[width=0.5\linewidth]{Images/hex-change-material.png}
		\caption{Changing the material to be a control rod.}
		\label{fig:Hex13}
	\end{center}
\end{figure}

Be sure to press apply to ensure your changes go into effect.

We should see a screen like Figure ~\ref{fig:Hex14} the after we're done.

\begin{figure}[H]
	\begin{center}
		\includegraphics[width=0.85\linewidth]{Images/hex-final-pin.png}
		\caption{After adding the cylinder.}
		\label{fig:Hex14}
	\end{center}
\end{figure}

\section{Populating assembly with our pin}

Next, we need to add this pin to our assembly lattice.  Make sure you select ``Assy\_0" in the inputs menu.  Click the lattice tab in the lower panel, and right click on the hexagon to bring up a list of available pins.  Select our pin (Figure ~\ref{fig:Hex18}).

\begin{figure}[H]
	\begin{center}
		\includegraphics[width=0.5\linewidth]{Images/hex-set-pin.png}
		\caption{Selecting our pin for the assembly.}
		\label{fig:Hex18}
	\end{center}
\end{figure}

Click the apply button to save your changes.

\section{Populating core with our assembly}

Click on our assembly (``Assy\_0") in the inputs pane.  In the lower pane, click on the lattice tab.  To add the assembly to any of the cells, right click to bring up a list of available assemblies, and then click on the assembly you'd like (Figure ~\ref{fig:Hex19}).

\begin{figure}[H]
	\begin{center}
		\includegraphics[width=0.5\linewidth]{Images/hex-assign-assy.png}
		\caption{Selecting assembly for a cell.}
		\label{fig:Hex19}
	\end{center}
\end{figure}

We'll select Assy\_0 because that's the name of the assembly we'd like.  In this case, we want to replace all remaining empty cells with Assy\_0, so we use the \ui{Replace All With}. Make sure to click the apply button to make sure that the changes we've made are applied.  Your screen should look like Figure  ~\ref{fig:Hex20} when you're done.

\begin{figure}[H]
	\begin{center}
		\includegraphics[width=0.85\linewidth]{Images/hex-final-core.png}
		\caption{Final view.}
		\label{fig:Hex20}
	\end{center}
\end{figure}

Congratulations!  You've made a hexagonal core!